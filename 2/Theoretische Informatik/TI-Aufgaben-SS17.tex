\documentclass{scrartcl}
\usepackage{amsmath,amssymb,amstext}
\usepackage[ngerman]{babel}
\usepackage{ucs}
\usepackage[utf8x]{inputenc}
\usepackage[T1]{fontenc}
\usepackage{enumitem}
\begin{document}
\begin{enumerate}
\item Formale Sprachen

\begin{enumerate}[label*=\arabic*.]
\item Sei $\Sigma = \{a, b, c,\}$ ein Alphabet.
\begin{enumerate}[label=(\alph*)]
\item Listen Sie alle Wörter über $\Sigma$ mit $|w| \leq 2$ auf.
\\ $\{\epsilon,a,b,c,aa,ab,ac,ba,bb,bc,ca,cb,cc\}$
\item Wie viele Sprachen $L \subseteq \{w | w \in \Sigma^*, |w| = 2 \}$ gibt es? Spezifizieren Sie drei davon.
\\ 512 Sprachen ($2^9$), 
\\ $\{\epsilon,a,b,c\}$, 
\\ $\{\epsilon,aa,ab,ac,ba,bb,bc,ca,cb,cc\}$, 
\\ $\{\epsilon,a,b,c,aa,ab,ac,ba,bb,bc,ca,cb,cc\}$
\item Definieren Sie drei unendliche Sprachen über $\Sigma$.
\\ $ L=\{w|w \in \Sigma^*\}$
\\ $L=\{w|w \in \Sigma^+\}$
\\ $L=\{w| |w| \geq 2\}$
\end{enumerate}

\item $L_1 = \{0^i | i \in \mathbb{N}$ und $L_2 = \{1^i | i \in \mathbb{N}_0\}$ seien formale Sprachen über dem Alphabet $\Sigma = \{0, 1\}$. Berechnen Sie:

\begin{enumerate}[label=(\alph*)]
\item $L_1 \cup L_2$
\\ $\{\epsilon, 0,1,00,11,...\}$
\item $L_1 \cap L_2$
\\ $\{\epsilon\}$
\item $L_1 \setminus L_2$
\\ $\{0,1,00,11,...\}$
\item $L_1 \cap \Sigma^*$
\\ $L_1$
\item $(L_1 \cup L_2) \cap \Sigma^3$
\\ $\{000,111\}$
\end{enumerate}

\item Sei $\Sigma = \{\$,\%,\&\}$ ein Alphabet.
\begin{enumerate}[label=(\alph*)]
\item Definieren Sie eine lineare Ordnung auf $\Sigma$.
\\ $\$ \prec \% \prec \&$
\item Listen sie alle Wörter $w$ über $\Sigma$ mit $|w| \leq 2$ in lexikographischer Ordnung bzgl. der unter (a) definierten linearen Ordnung auf.
\\ $\{\$ \prec \% \prec \& \prec \$\$ \prec \$\% \prec \$\& \prec \%\$ \prec \%\% \prec \%\& \prec \&\$ \prec \&\% \prec \&\&\}$
\item Welche Wörter gehören zur Sprache $L = \{w | w \in \bigcup\limits_{i=0}^2 \Sigma^i, w = w^R\}$?
\\ $\{\epsilon,\$,\%,\&,\$\$,\%\%,\&\&\}$
\end{enumerate}

\pagebreak

\item Sei $\Sigma$ ein Alphabet aus $n$ Zeichen, $n \in \mathbb{N}$.
\begin{enumerate}[label=(\alph*)]
\item Wie viele Wörter enthält $\Sigma^m, m \in \mathbb{N}_0$?
\\ unendlich
\item Wie viele Wörter enthält $\bigcup\limits_{i=0}^{m} \Sigma^i, m \in \mathbb{N}_0$?
\\ unendlich
\item Wie viele Wörter enthält $\Sigma^*$?
\\ unendlich
\end{enumerate}
\end{enumerate}
\end{enumerate}

\pagebreak

\begin{enumerate}
\item
\begin{enumerate}[label=(\alph*)]
\item 
Sei $\Sigma = \{+,\&,\#\}$ ein Alphabet, auf dem eine lineare Ordnung wie folgt definiert ist:
\\ $\# \preceq_\Sigma \& \preceq_\Sigma +$ \\
Bestimmen Sie die Sprache $\Sigma^0 \cup \Sigma^1 \cup \Sigma^2$ und listen Sie die darin enthaltenen Wörter in Wortordnung auf.
\item Es sei $L = \{0^{2i+1} | i \in \mathbb{N}_0\}$ eine Sprache uber dem Alphabet $\Sigma = \{0\}$.
\\ Bestimmen Sie $\Sigma^+ \setminus L$
\item  Es seien $\Sigma$ ein Alphabet mit $|\Sigma| = 5$ und $k \in \mathbb{N}_0$.
Bestimmen Sie $|\Sigma^k|$.
\end{enumerate}
\end{enumerate}

Lösung:
\begin{enumerate}[label=(\alph*)]
\item $L= \{\epsilon \preceq_\Sigma \# \preceq_\Sigma \& \preceq_\Sigma + \preceq_\Sigma \#\# \preceq_\Sigma \#\& \preceq_\Sigma \#+ \preceq_\Sigma \&\# \preceq_\Sigma \&\& \preceq_\Sigma \&+ \preceq_\Sigma +\# \preceq_\Sigma +\& \preceq_\Sigma ++\}$
\item $L = \{0²i | i \in \mathbb{N}_0\}$
\item $5^k$
\end{enumerate}

\end{document}