\documentclass{article} 

\usepackage[utf8]{inputenc} % set input encoding (not needed with XeLaTeX)

\usepackage{geometry}
\geometry{margin=3cm}

\usepackage{forest}
\usepackage{mathtools}

\usepackage{titlesec}
\titlespacing{\subsection}{0pt}{*6}{*1.5}
\renewcommand{\thesubsection}{\arabic{subsection}.}

\begin{document}
\title{\Large Einsendeaufgabe 1}
\author{\normalsize Stefan Berger}
\date{}
\maketitle

\subsection{}
Der Algorithmus werde mit TuermeHanoi(3; A; B; C) aufgerufen. Zeigen Sie den Ablauf des
Algorithmus mit einem Rekursionsbaum.\\\\

\begin{forest}
[TH(3; A; B; C)
	[TH(2; A; C; B) 
		[TH(1; A; B; C)
			[ZS(A; C)]
		]
		[ZS(A; B), before computing xy={s/.average={s}{siblings}}]
		[TH(1; C; A; B)
			[ZS(C; B)]
		]
	] 
	[ZS(A; C), before computing xy={s/.average={s}{siblings}}]
	[TH(2; B; A; C)
		[TH(1; B; C; A)
			[ZS(B; A)]
		]
		[ZS(B; C), before computing xy={s/.average={s}{siblings}}]
		[TH(1; A; B; C
			[ZS(A; C)]
		]
	]
]
\end{forest}

\stepcounter{subsection}
\subsection{}
Wie viele Aufrufe gibt es für n=3, n=4, n=5? \\\\
\begin{tabular}{|c|c|c|c|}
\hline
n & 3 & 4 & 5 \\
\hline
rek. Aufrufe & 7 & 15 & 31 \\
\hline
\end{tabular}

\subsection{}
Verallgemeinern Sie Ihr Ergebnis aus Frage 3 für n in N. Schauen Sie die Tabelle im Skript Seite
23 an. Wie wächst die Funktion, die Sie gefunden haben, im Vergleich? \\\\
\begin{math}
g(n) \in \Theta(2^n - 1)
\end{math}
\\
Der Algorithmus ist nicht effizient.

\end{document}
