\documentclass{article} % use larger type; default would be 10pt

\usepackage[utf8]{inputenc} % set input encoding (not needed with XeLaTeX)

%%% Examples of Article customizations
% These packages are optional, depending whether you want the features they provide.
% See the LaTeX Companion or other references for full information.


\usepackage{geometry}
\geometry{margin=3cm}

\usepackage[fleqn]{amsmath}
\usepackage{amssymb}

\usepackage{hyperref}
\hypersetup{colorlinks=true, allcolors=blue}

\usepackage{titlesec}
\titlespacing{\subsection}{0pt}{*6}{*1.5}
\renewcommand{\thesubsection}{\arabic{subsection}.}

\usepackage[plain]{algorithm}
\usepackage{algpseudocode}
\algtext*{EndWhile}
\algtext*{EndIf}

\begin{document}
\title{\Large Einsendeaufgabe 2}
\author{\normalsize Stefan Berger}
\date{}
\maketitle

\subsection{}
Schreiben Sie einen rekursiven Algorithmus für die binäre Suche Seite 43 im Skript. Siehe auch
Übung 3 Übungsaufgaben Datei AlgOnl-Aufg-WS17-18-oL.pdf im Kursmaterial.\\

\begin{algorithm}
\textbf{Eingabe:} $A=(a_1, ..., a_n)$ mit $a_i, i, n \in \mathbb{N}, 1 \leq i \leq n, x \in \mathbb{N}, a_1<a_2<...<a_n$\\
\textbf{Ausgabe:} $i \in {0, 1, ..., n}$ mit $1 \leq i \leq n$ gefunden, $i=0$ nicht gefunden\\
\begin{algorithmic}[1]

\State \Call {Binares Suchen}{$A, x, 1,$ Länge von $A$ + 1}
\Function{Binares Suchen}{$A, x, l, r$}
	\If {$l\geq r$}
	\State \Return 0
	\EndIf
	\State $i\gets \lfloor{(l+r)/2} \rfloor$
	\If {$A[i]=x$}
	\State \Return {$i$}
	\EndIf
	\If {$A[i] < x$}
	\State $l\gets i+1$
	\Else
	\State $r\gets i$
	\EndIf
	\State \Return \Call {Binares Suchen}{$A, x, l, r$}
\EndFunction
\end{algorithmic}
\end{algorithm}
\stepcounter{subsection}
\subsection{}
Erzeugen Sie ein int Array mit den Zahlen 0 bis 999 und suchen Sie 12. Wie viele Aufrufe sind
notwendig? Gleiche Frage mit Zahlen von 0 bis 99999.\\

Der Algorithmus benötigt 10 rekursive Aufrufe, um die Zahl 12 in dem Array mit den Zahlen von 0 bis 999 zu finden. Für das 
Array mit den Zahlen 0 bis 99999 benötigt er 16 rekursive Aufrufe. Der Algorithmus hat eine logarithmische Komplexität ($T(n) = \lfloor{\log n + 1}\rfloor$).

\pagebreak
\subsection{}
Stellen Sie die Rekursionsgleichung zur Bestimmung der Zeitkomplexität Ihres Algorithmus im
schlechtesten Fall in Abhängigkeit von der Eingabe n auf.\\

\begin{math}
T(n) =
\begin{cases}
konstant, & falls\ n = 1\\
T(n / 2) + \Theta(1), & falls\ n \geq 2
\end{cases}
\end{math}

\subsection{}
Lösen Sie die Rekursionsgleichung mit Hilfe des Master-Theorems und geben sie die Zeit-
Komplexität in asymptotischer Notation. Entspricht das Ergebnis Ihrer Antworten der Frage 3?

\begin{math}\\
b=1, c = 2, k = 0 \\\\
b(1/c)^k = (1/2)^0 = 1 \\\\
T(n) \in \Theta(\log n) \\\\
\end{math}
\end{document}
